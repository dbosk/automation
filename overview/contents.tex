\mode*

\section{The problem}

\begin{frame}[fragile]
  \begin{block}{The problem}
    \begin{description}
      \item<+>[What's the problem?] Simple tasks that take up our time.
      \item<+>[Why is it a problem?] We waste time that can be used better.
      \item<+>[What's the approach?] Let a computer do these things instead.
      \item<+>[What are the findings?] This works very well.
    \end{description}
  \end{block}
\end{frame}

\subsection{Various simple tasks that take time}

Det vore önskvärt att få ta del av kollegors lösningar på:

\begin{frame}[fragile]\label<2>{WishGradingReporting}
  \begin{example}[Grading]
    \begin{itemize}
      \item Grade/comment assignments when there are several teachers
      \item \alert<2>{Reporting to Canvas when connected to automated grading 
        (\eg gits-15, kattis, mail, zoom participation \etc)}
      \item Grading rubrics (betygsmatriser)
      \item \alert<2>{Grading/commenting assignments}
    \end{itemize}
  \end{example}

  \begin{example}[Reporting]
    \begin{itemize}
      \item \alert<2>{Bulk reporting}
      \item \alert<2>{Formulas for grading}
      \item \alert<2>{Canvas to LADOK with correct dates}
      \item \alert<2>{Automated grading + reporting}
    \end{itemize}
  \end{example}
\end{frame}

\begin{frame}[fragile]\label<2>{WishLateStudents}
  \begin{example}[Late students]
    \begin{itemize}
      \item \alert<2>{Discover finished course or components}
      \item \alert<2>{Grade students from different course rounds}\footnote{%
          More difficult with locked Canvas rooms.
        }:
        \begin{description}
          \item[Scenario 1] \alert<2>{Get email from student who claims they've 
            finished but can't see any result in LADOK.}

          \item[Scenario 2] The student took the exam during another round than 
            what they originally registered for.
        \end{description}

      \item \alert<2>{It's more work to track students from previous rounds, 
        what they've finished, bonus points etc.}
    \end{itemize}
  \end{example}
\end{frame}

\begin{frame}[fragile]
  \begin{example}[Managing students]
    \begin{itemize}
      \item Time booking
      \item Group formation
      \item Manage course related emails or Canvas messages
      \item Manage discussion threads
    \end{itemize}
  \end{example}

  \pause

  \begin{remark}
    \begin{itemize}
      \item Won't cover any of these topics.
      \item But if you want to talk about it, contact us.
    \end{itemize}
  \end{remark}
\end{frame}

\begin{frame}[fragile]\label<2>{WishTA}
  \begin{example}[Managing TAs]
    \begin{itemize}
      \item \alert<2>{Time booking}

      \item \alert<2>{Find potential TAs, the TA pool}
        \begin{itemize}
          \item the TA pool doesn't contain details of which course an 
            individual is suitable to work on.
        \end{itemize}

      \item \alert<2>{Remove students who graduated from the TA pool.}
    \end{itemize}
  \end{example}
\end{frame}


\section{Degree projects}

I would like to take 15 minutes to talk about Canvas, the thesis templates      
(and documentation), oral presentations with active listeners, covers, and      
the reporting in LADOK and DiVA.                                                
                                                                                
The focus is that every degree project leads to an entry in LADOK and those     
that are successful lead to a thesis and the meta data being put into DiVA.     
By putting some information into the thesis this reporting can be               
facilitated. Additionally, by putting some information into the Canvas          
course room, one can facilitate the entire degree project process, create a     
customize template for the student (or students), easily create the             
announcement for the oral presentation, handle reporting of active              
listening, make a correct cover, produce the meta data for DiVA and the         
reporting (with titles) into LADOK.

\mode<all>{\includepdf[pages=2-7]{./degree-projects-20221004.pdf}}


\section[Grading]{Grading: \texttt{canvaslms}}

\begin{frame}[fragile]
  \begin{example}[Grading]
    \begin{itemize}
      \item Grade/comment assignments when there are several teachers
      \item \alert{Reporting to Canvas when connected to automated grading (\eg 
        gits-15, kattis, mail, zoom participation \etc)}
      \item Grading rubrics (betygsmatriser)
      \item \alert{Grading/commenting assignments}
    \end{itemize}
  \end{example}
\end{frame}

\begin{frame}
  \begin{example}[Late students]
    \begin{itemize}
      \item \alert<1>{Discover finished course or components}
      \item \alert<1-2>{Grade students from different course rounds}\footnote{%
          More difficult with locked Canvas rooms.
        }:
        \begin{description}
          \item[Scenario 1] \alert<1>{Get email from student who claims they've 
            finished but can't see any result in LADOK.}

          \item[Scenario 2] The student took the exam during another round than 
            what they originally registered for.
        \end{description}

      \item \alert<1>{It's more work to track students from previous rounds, 
        what they've finished, bonus points etc.}
    \end{itemize}
  \end{example}
\end{frame}

\subsection{DD1301 Computer Introduction, 1.5 credits}

\begin{frame}[fragile]
  \begin{block}{Learning Outcomes}
    \begin{itemize}
      \item<1> obtain access to the computational environment at KTH
      \item<2> control the computer environment via the command line
      \item<3> handle submissions of program code with version control tools
      \item<4> create and compile technical reports
    \end{itemize}
  \end{block}

  \begin{block}{Assignments}
    \begin{itemize}
      \item<1> SSH: \mintinline{bash}{ssh student-shell.sys.kth.se}
      \item<2> Terminal: \mintinline{bash}{history > ~/Public/datintro/history.txt}
      \item<3> Git: Make any update to a given repo.
      \item<4> LaTeX: Write a LaTeX document with title, sections, references, 
        etc.
    \end{itemize}
  \end{block}
\end{frame}

\begin{frame}[fragile]
  \begin{example}[Grading]
    \begin{itemize}
      \item Run grading scripts on a Raspberry Pi.
      \item Uses
        \begin{itemize}
          \item RepoBee
          \item canvaslms
        \end{itemize}
    \end{itemize}
  \end{example}
\end{frame}

\begin{frame}[fragile]
  \begin{block}{RepoBee}
    \begin{itemize}
      \item Simon Larsen and Ric Glassey @ EECS TCS
      \item Works with GitHub (gits-15).
      \item \url{https://repobee.org}
      \item \url{https://github.com/repobee/repobee}
    \end{itemize}
  \end{block}

  \begin{example}[Cloning all students' repos]
    \begin{minted}{bash}
repobee repos clone -qq \
    --sf $students -o $org --discover-repos
    \end{minted}
  \end{example}
\end{frame}

\begin{frame}[fragile]
  \begin{block}{\texttt{canvaslms}}
    \begin{itemize}
      \item Daniel Bosk @ EECS TCS
      \item Command-line interface for Canvas
      \item \url{https://github.com/dbosk/canvaslms}
    \end{itemize}
  \end{block}

  \begin{example}[\texttt{canvaslms} uses]
    \begin{itemize}
      \item \mintinline{bash}{canvaslms users -sc datintro22}
      \item \mintinline{bash}{canvaslms submissions -c datintro22 -a LaTeX}
    \end{itemize}
  \end{example}
\end{frame}

\begin{frame}[fragile]
  % workaround from 
  % https://github.com/gpoore/minted/issues/70#issuecomment-111729930
  \begingroup
  \catcode §=\active
  \def§#1§{\only<1-2>{\alert<2>{#1}}\only<3>{}}
  \begin{minted}[fontsize=\Large,escapeinside=||]{bash}
    canvaslms grade -c datintro|§22§| -a "terminal" \
      -u student@kth.se -g P -m "Well done!"
  \end{minted}
  \endgroup
\end{frame}

\begin{frame}[fragile]
  \begin{example}[Setting up DD1301]
    \begin{minted}{bash}
      docker run \
        -e KRB_USER -e KRB_PASS \
        -e CANVAS_SERVER -e CANVAS_TOKEN \
        -e REPOBEE_USER -e REPOBEE_URL -e REPOBEE_TOKEN \
        -e COURSE_CODE -e COURSE_CODE_ORG \
          datintro22-setup:latest
    \end{minted}
  \end{example}

  \pause

  \begin{remark}[Tools]
    \begin{itemize}
      \item \mintinline{bash}{canvaslms users -s}
      \item \mintinline{bash}{repobee repos setup}
    \end{itemize}
  \end{remark}
\end{frame}

\begin{frame}[fragile]
  \begin{example}[Keeping students up-to-date on grading]
    \begin{minted}{bash}
      # https://people.kth.se/~dbosk/datintro/grader.log
      LOG_FILE="/afs/kth.se/home/d/b/dbosk/public_html/datintro/grader.log"
      mkdir -p $(dirname $LOG_FILE)
      echo -n "Started grading $(date) " >> $LOG_FILE

      # do grading

      echo "-> Finished $(date)" >> $LOG_FILE
    \end{minted}
  \end{example}
\end{frame}

\begin{frame}[fragile]
\begin{minted}{bash}
out="$(docker run \
  -e KRB_USER -e KRB_PASS \
  -e CANVAS_SERVER -e CANVAS_TOKEN \
  -e REPOBEE_USER -e REPOBEE_URL -e REPOBEE_TOKEN \
  -e COURSE_CODE -e COURSE_CODE_ORG \
  -v /afs:/afs \
    datintro22-grade:latest)"
if [ -n "$out" ]; then
  echo "$out"
  echo
  echo "LADOK:"
  canvaslms results -c "$courses" -A "$components" | \
    sed -E "s/ ?[HV]T[0-9]*[^\t]*//" | \
    ladok report -fv
fi

\end{minted}
\end{frame}


\section[Grade reporting]{Grade reporting: \texttt{canvaslms results | ladok}}

- I talk about automation of normal courses, reporting labs. This covers  
points:

    1 bulkrapportering,
    2 upptäcka avslutad kurs eller komponenter,
    3 Canvas till Ladok med rätt datum,
    4 formler för betygsättning,
    9 hantering av csv export (automatiserad)
    12 rubriker (rättningsmatriser?)
    13 inrapportering i Canvas kopplad till automatisk betygsättning
    14 automatiserad betygssättning/resultatrapportering
    17 betygsätta studenter från olika kursutbud

\begin{frame}[fragile]
  \begin{example}[Reporting]
    \begin{itemize}
      \item \alert{Bulk reporting}
      \item \alert{Formulas for grading}
      \item \alert{Canvas to LADOK with correct dates}
      \item \alert{Automated grading + reporting}
    \end{itemize}
  \end{example}
\end{frame}

\begin{frame}
  \begin{example}[Late students]
    \begin{itemize}
      \item \alert{Discover finished course or components}
      \item \alert{Grade students from different course rounds}\footnote{%
          More difficult with locked Canvas rooms.
        }:
        \begin{description}
          \item[Scenario 1] \alert{Get email from student who claims they've 
            finished but can't see any result in LADOK.}

          \item[Scenario 2] The student took the exam during another round than 
            what they originally registered for.
        \end{description}

      \item \alert{It's more work to track students from previous rounds, what 
        they've finished, bonus points etc.}
    \end{itemize}
  \end{example}
\end{frame}

\begin{frame}[fragile]
  \begin{block}{\texttt{ladok3}}
    \begin{itemize}
      \item Daniel Bosk, Alexander Baltatzis and Chip Maguire
      \item Python package and command-line interface for LADOK3
      \item \url{https://github.com/dbosk/ladok3}
    \end{itemize}
  \end{block}

  \begin{example}[\texttt{ladok3} uses]
    \begin{itemize}
      \item \mintinline{bash}{ladok report}
      \item \mintinline{bash}{ladok data}

        \pause

      \item \mintinline{python}{import ladok3}
    \end{itemize}
  \end{example}
\end{frame}

\subsection{DD1301 Computer Introduction, 1.5 credits}

\begin{frame}[fragile]
  \begingroup
  \catcode §=\active
  \def§#1§{\only<1-2>{\alert<2>{#1}}\only<3>{}}
  \begin{minted}[fontsize=\Large,escapeinside=@@]{bash}
    canvaslms results -c datintro@§22§@ -A LAB1 | \
      sed -E "s/ ?[HV]T[0-9]*[^\t]*//" | \
      ladok report -f
  \end{minted}
  \endgroup
\end{frame}

\subsection{DD1310 + DD1315 + DD1317 Programming Techniques}

\begin{frame}[fragile]
  \begin{minted}[fontsize=\Large]{bash}
    canvaslms results -c prgm22 \
      -A "LAB[1-3]" | \
        sed -E "s/ ?[HV]T[0-9]*[^\t]*//" | \
          ladok report -f
  \end{minted}

  \begin{example}[Matching]
    \begin{itemize}
      \item prgm22
      \item LAB1, LAB2, LAB3
    \end{itemize}
  \end{example}
\end{frame}

\begin{frame}[fragile]
  \begin{minted}[fontsize=\Large]{bash}
    canvaslms results -c prgi22 \
      -A "(LAB[1-3]|KAL1)" | \
        sed -E "s/ ?[HV]T[0-9]*[^\t]*//" | \
          ladok report -f
  \end{minted}

  \begin{example}[Matching]
    \begin{itemize}
      \item prgi22
      \item LAB1, LAB2, LAB3, KAL1
    \end{itemize}
  \end{example}
\end{frame}

\begin{frame}[fragile]
  \begin{minted}[fontsize=\Large]{bash}
    canvaslms results -c "prg[im]" \
      -A "(LAB[1-3]|KAL1|MAT1)" | \
        sed -E "s/ ?[HV]T[0-9]*[^\t]*//" | \
          ladok report -f
  \end{minted}

  \begin{example}[Matching]
    \begin{itemize}
      \item DD1310 prgm2x: LAB1, LAB2, LAB3
      \item DD1317 prgi22: LAB1, LAB2, LAB3, KAL1
      \item DD1315 prgi21: LAB1, LAB2, LAB3, MAT1
    \end{itemize}
  \end{example}
\end{frame}

\subsection{Formulas for grades}

\begin{frame}[fragile]
  \begin{question}
    \begin{itemize}
      \item How do we know how to merge several assignments into one grade in 
        LADOK?
    \end{itemize}
  \end{question}

  \pause
  
  \begin{solution}
  \begin{minted}[fontsize=\Large]{bash}
    canvaslms results -h
    pydoc3 canvaslms.grades
  \end{minted}
  \end{solution}
\end{frame}

\subsection{Across course rounds}

\begin{frame}
  \begin{example}[Late students]
    \begin{itemize}
      \item \alert<1>{Discover finished course or components}
      \item \alert<1>{Grade students from different course rounds}\footnote{%
          More difficult with locked Canvas rooms.
        }:
        \begin{description}
          \item[Scenario 1] \alert<1>{Get email from student who claims they've 
            finished but can't see any result in LADOK.}

          \item[Scenario 2] The student took the exam during another round than 
            what they originally registered for.
        \end{description}

      \item \alert<1>{It's more work to track students from previous rounds, 
        what they've finished, bonus points etc.}
    \end{itemize}
  \end{example}
\end{frame}

\begin{frame}[fragile]
  \begin{block}{The cases}
    \begin{enumerate}
      \item A student has finished everything related to some LADOK modules, 
        but not all.
      \item A student has finished some assignments related to LADOK module X, 
        but not all.
    \end{enumerate}
  \end{block}

  \begin{question}
    \begin{itemize}
      \item How do we handle these cases?
    \end{itemize}
  \end{question}
\end{frame}

\begin{frame}[fragile]
  \begin{block}{The cases}
    \begin{enumerate}
      \item A student has finished everything related to some LADOK modules, 
        but not all.
      \item A student has finished some assignments related to LADOK module X, 
        but not all.
    \end{enumerate}
  \end{block}

  \begin{solution}[Case 1]
    \begin{itemize}
      \item Our reporting above works.
    \end{itemize}
  \end{solution}
\end{frame}

\begin{frame}[fragile]
  \begin{block}{The cases}
    \begin{enumerate}
      \item A student has finished everything related to some LADOK modules, 
        but not all.
      \item A student has finished some assignments related to LADOK module X, 
        but not all.
    \end{enumerate}
  \end{block}

  \begin{solution}[Case 2]
    \begin{itemize}
      \item Within a year: Restlabbar and Lab Week!
      \item Otherwise
    \end{itemize}
    \begin{minted}{bash}
      canvaslms submissions -c DD1310 -a Lab \
        -u student@kth.se | \
          sort -t $'\t' -k 2 > labs-sorted.tsv
    \end{minted}
  \end{solution}
\end{frame}

\section[TA management]{TA management: \texttt{nytid}}

- I can also talk about some automation of TA management, some upcoming
work on recruiting TAs, a TA pool etc. This covers points:

5 gruppbildning
6 tidsbokning
15 hitta potentiella TA:er (amanuenser som har gått din kurs)
16 ta bort studenter som tagit examen från TA-poolen


Alexander can also add some things on this topic, how he transferred
student bookings from Canvas calendar into the queuing system.

\begin{frame}[fragile]\label<2>{WishTA}
  \begin{example}[Managing TAs]
    \begin{itemize}
      \item \alert{Time booking}

      \item \alert{Find potential TAs, the TA pool}
        \begin{itemize}
          \item the TA pool doesn't contain details of which course an 
            individual is suitable to work on.
        \end{itemize}

      \item \alert{Remove students who graduated from the TA pool.}
    \end{itemize}
  \end{example}

  \pause

  \begin{example}[Bonus!]
   \begin{itemize}
     \item How Daniel doesn't have to review and sign >70 sheets every month!
   \end{itemize} \end{example}
\end{frame}

\begin{frame}[fragile]
  \begin{block}{\texttt{nytid}}
    \begin{itemize}
      \item Daniel Bosk
      \item Python package and \emph{soon} CLI to manage TAs' time
      \item \url{https://github.com/dbosk/nytid}
    \end{itemize}
  \end{block}

  \begin{example}[\texttt{nytid} uses]
    \begin{itemize}
      \item TAs sign up for lab sessions.
      \item Automatically generate time sheets.
      \item Report new TAs to HR.
      \item Automatically request amanuensis contracts.

        \pause

      \item \mintinline{python}{import nytid}
    \end{itemize}
  \end{example}
\end{frame}

\subsection{Future uses}

\begin{frame}[fragile]
  \begin{example}[TA pool]
    \begin{itemize}
      \item \mintinline{bash}{nytid recommend -c "DD131[0-9]" -u student@kth.se}
      \item \mintinline{bash}{nytid reqruit -c "prgi23"}
      \item Can use \texttt{ladok3} to filter out graduated students.
    \end{itemize}
  \end{example}
  
  \pause

  \begin{example}[TA sign up]
    \begin{itemize}
      \item \mintinline{bash}{nytid schedule mine --set  https://timeedit...}
      \item \mintinline{bash}{nytid signup -c "prgi23"}
      \item \mintinline{bash}{nytid schedule signed --set ~/public_html/TA.ics}
    \end{itemize}
  \end{example}
\end{frame}

\begin{frame}[fragile]
  \begin{example}[TA confirmation]
    \begin{itemize}
      \item \mintinline{bash}{nytid checkin -c prgi23 --next }
      \item TimeEdit to automate running check-in.
      \item Email or Zulip (or Slack) API to send reminders to check in.
      \item \mintinline{bash}{nytid confirm -c prgi23 --now TA1 TA2 TA3}
      \item \mintinline{bash}{nytid confirm -c prgi23 --time "2023-09-31 13:15"}
    \end{itemize}
  \end{example}

  \pause

  \begin{example}[Reporting hours]
    \begin{itemize}
      \item \mintinline{bash}{nytid report}
    \end{itemize}
  \end{example}
\end{frame}

\section{Tools and APIs}

\begin{frame}[fragile]
  \begin{center}
    \LARGE
    KTH giveth and KTH taketh away \dots
  \end{center}

  \begin{remark}
    \begin{itemize}
      \item KTH removes lifetime email addresses
      \item KTH removes personal web pages \alert<+>{(useful for these tools)}
      \item KTH almost removed (maybe still will) GitHub Enterprise
      \item KTH removes KTH Social schedule API
      \item \alert<+>{KTH will introduce MFA, but probably not aligned with our 
        work}
      \item \dots
    \end{itemize}
  \end{remark}
\end{frame}

\begin{frame}[fragile]
  \begin{block}{Development approach}
    \begin{itemize}
      \item Open source
      \item Independent from institution
    \end{itemize}
  \end{block}

  \pause

  \begin{example}[Advantages]
    \begin{itemize}
      \item Teachers at other institutions can use the tools
      \item More people can participate in maintenance and 
        development
    \end{itemize}
  \end{example}
\end{frame}

\begin{frame}
  \begin{example}[RepoBee]
    \begin{itemize}
      \item Developed with talented amanuensis.
      \item Yielded six (6) papers at ACM CS Ed conferences.
      \item Most active collaborators at TU Eindhoven, but also other 
        universities.
      \item And of course many anonymous users.
      \item Now self-supporting OSS project.
    \end{itemize}
  \end{example}

  \pause

  \begin{example}[\texttt{ladok3}]
    \begin{itemize}
      \item Invited to Mid Sweden University to talk about these tools.
    \end{itemize}
  \end{example}
\end{frame}

\begin{frame}[fragile]
  \begin{remark}[Gulan et al.~were wrong]
    \begin{itemize}
      \item So yes, I'd argue that Gulan et al.~were wrong about developing our 
        own tools.
      \item Better to build useful tools ourselves, than buying useless crap.
    \end{itemize}
  \end{remark}
\end{frame}

\begin{frame}[fragile]
  \begin{example}[Installation]
    \begin{itemize}
      \item \mintinline{bash}{python3 -m pip install canvaslms ladok3 nytid}
    \end{itemize}
  \end{example}

  \pause

  \begin{block}{Development}
    \begin{itemize}
      \item \url{https://github.com/dbosk/canvaslms}
      \item \url{https://github.com/dbosk/ladok3}
      \item \url{https://github.com/dbosk/nytid}
    \end{itemize}
  \end{block}
\end{frame}

\subsection{What do we need to build these tools?}

\begin{frame}[fragile]
  \begin{example}[Some things we need]
    \begin{itemize}
      \item Time \dots
      \item \alert<2>{APIs}
    \end{itemize}
  \end{example}
\end{frame}

- Alexander can talk about that there is an embryo of IT-support
(REST-api) for this kind of things. He can present where we are at the
moment and some options for going forward.

We can have a discussion at the end. I have some opinions in this area
too.

